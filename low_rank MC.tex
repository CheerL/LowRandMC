\documentclass[UTF8]{ctexart}
%\documentclass{article}

\title{低秩矩阵完整化问题的几种高效解法}
\author{林陈冉}
\date{\today}

\usepackage{geometry}
\usepackage{amsmath}
\usepackage{amssymb}
\geometry{papersize={21cm,29.7cm}}
\geometry{left=1.91cm,right=1.91cm,top=2.54cm,bottom=2.54cm}

\newtheorem{theo}{定理}[section]
\newtheorem{define}{定义}[section]
\newtheorem{algo}{算法}

\newcommand{\s}{\quad}
\renewcommand{\b}{\textbf}
\newcommand{\p}{\paragraph{}\s}
\renewcommand{\sp}{\subparagraph}
\newcommand{\sect}{\section}
\newcommand{\ssect}{\subsection}
\newcommand{\sssect}{\subsubsection}
\newcommand{\equSplit}[1]{\begin{equation}\begin{split}#1\end{split}\end{equation}}
\newcommand{\equAlign}[1]{\begin{align}#1\end{align}}
\newcommand{\equ}[1]{\begin{equation}#1\end{equation}}
\newcommand{\Tst}{\text{s.t.}\s}
\newcommand{\abs}[1]{\lvert#1\rvert}
\newcommand{\norm}[1]{\lVert#1\rVert}
\newcommand{\inprod}[1]{\langle#1\rangle}
\newcommand{\Real}[1]{\mathbb{R}^{#1}}
\newcommand{\nunorm}{\norm{X}_*}
\newcommand{\Ma}{\mathcal{A}}
\newcommand{\partD}[2]{\frac{\partial#1}{\partial#2}}
\newcommand{\pMa}[1]{\begin{pmatrix}#1\end{pmatrix}}
\newcommand{\vMa}[1]{\begin{vmatrix}#1\end{vmatrix}}
\newcommand{\bMa}[1]{\begin{bmatrix}#1\end{bmatrix}}

\renewcommand{\theequation}{\thesection.\arabic{equation}}
\numberwithin{equation}{section}

\begin{document}
\maketitle
\section{简介}
\ssect{什么是低秩矩阵完整化(low-rank matrix completion)}

\p简单来说,低秩矩阵完整化问题,就是在仅仅知道矩阵的少部分元素的情况下,恢复出这个矩阵的所有元素.这个问题在统计,图像处理,计算几何,机器学习,信号处理,模型控制等方面有广泛应用,比如著名的NetFlix大奖赛问题.

\ssect{数学模型}

\p显而易见,补全一个完全随机的矩阵几乎是不可能的,也是意义不大的.一般情况下,我们认为所需要补全的矩阵是有一定规律的,也就是说,这个矩阵的秩比较小.通常我们最感兴趣的是这样的一个最优化问题:
\equSplit{
	&\min rank(X)\\
	&\Tst X_{ij}=M_{ij},\forall(i,j)\in\Omega\\
}
其中$X,M\in\Real{p\times q}$,$\Omega$是已知元素的下标$(i,j)$构成的集合,$\abs{\Omega}=m$.

\p在一些情况下,(1.1)等价于线性约束问题:
\equSplit{
	&\min\s rank(X)\\
	&\Tst \mathcal{A}(X)=b\\
}
其中$b=(b_1,\cdots,b_m)\in\Real{m}$,
$\mathcal{A}:\Real{p\times q}\rightarrow\Real{m}$是线性算子,
$\Ma(X)=(\inprod{A_1,X},\cdots,\inprod{A_m,X})$,
$A_i\in\Real{p\times q},\forall i=1,\cdots,m$,
$\inprod{A,X}$是矩阵内积,
\p在此给出线性算子$\Ma$的共轭算子的定义:
$\Ma^*:\Real{m}\rightarrow\Real{p\times q}$,
$\Ma^*(y)=\sum^m_{i=1}y_iA_i,\forall y=(y_1,\cdots,y_m)\in\Real{m}$.
容易验证$\Ma$和$\Ma^*$是well-defined,即
\[
	\inprod{\Ma(X),y}=\inprod{(\inprod{A_i,X}),(y_i)}=
	\sum^m_{i=1}y_i\inprod{A_i,X}=\inprod{\sum^m_{i=1}y_iA_i,X}
	=\inprod{\Ma^*(y),X}=\inprod{X,\Ma^*(y)}
\]

\p但是(1.1),(1.2)都是\b{"NP-难"}的,因此需要一定的转化.这里我们用核范数来近似矩阵的秩,把(1.1)与(1.2)转化为以下形式:

\equSplit{
	&\min\s\nunorm\\
	&\Tst \mathcal{A}(X)=b\\	
}

\equSplit{
	&\min\s\nunorm\\
	&\Tst X_{ij}=M_{ij},\forall(i,j)\in\Omega\\
}

\p很自然的,一个重要的问题是:(1.1)与(1.3),或者(1.2)与(1.4)什么时候等价?略过证明,直接描述以下重要的结论:
\p对于(1.3),Candes和陶哲轩等给出了一个证明.当在某些条件下,若已知元素个数$\abs{\Omega}=m=O(nr\cdot\text{polylog}(n))$,其中$n=max(p,q)$,polylog是多重对数函数,则矩阵有很高概率可以通过(1.3)恢复.
\p对于(1.4),Recht等给出了一个证明.将线性映射$\mathcal{A}$的矩阵形式记作$A$,即$\mathcal{A}(X)=A\text{vec}(X)$,其中$\text{vec}(X)\in\Real{pq}$为矩阵$X$的向量化.当$A$是一个随机高斯矩阵,若向量$b$的维数$m\geq C(r(p+q)log(pq)$,其中$C$是一个正的常数,则矩阵有很高概率可以通过(1.4)恢复.
\p实际上,由线性代数知识容易知道,(1.4)要求的线性约束条件并不总能成立,因此有时需要适当松弛.考虑(1.4)的罚函数:
\equ{\min\s\nunorm+\frac{1}{2\mu}\norm{\mathcal{A}(X)-b}_2^2}
其中$\mu$是某个给定常数.
\p下面,将对(1.3),(1.4),(1.5)分别给出一种高效的解法.

\sect{交替方向增广Lagrange法}
%\p对于问题(1.4),我们考虑它的Lagrange函数
%\equ{L(x,y)=\nunorm+y^\top(\Ma(X)-b),y\in\Real{m}}
\p对于问题(1.3),我们考虑它的对偶问题
\equSplit{
	\max_{y\in\Real{m}}\s&b^\top y\\
	\Tst&\norm{\Ma^*(y)}_2\leq1\\
}

\p引入一个形式上的变量$S$,将(2.1)变为如下的等价形式
\equSplit{
	\min_{y\in\Real{m}}\s&-b^\top y\\
	\Tst&\Ma^*(y)-S=0\\
	&\norm{S}_2\leq1\\
}

\p可以考虑(2.2)的增广Lagrange函数
\equ{L(y,S,X,\mu)=-b^\top y+\inprod{X,\Ma^*(y)-S}+\frac{1}{2\mu}\norm{\Ma^*(y)-S}^2_F}
其中$X\in\Real{p\times q}$是某个给定的矩阵(不同于原问题中的$X$),$\norm{\cdot}_F$是矩阵的\b{$F$-范数},定义为矩阵所有元素的平方和的平方根.

\p由此我们得到(2.1)的一个等价问题
\equSplit{
	\min_{y,S}\s&L(y,S,X,\mu)\\
	\Tst&\norm{S}_2\leq1\\
}
通过选取最佳的$\mu$和$X$可以求出(2.4)的最优解.

\p采用如下的迭代过程以求解问题(2.4)
\equAlign{
	&(y^{k+1},S^{k+1})=\arg\min_{y,S}L(y,S,X^k,\mu^k)\\
	&\mu^{k+1}\in[\alpha\mu^k,\mu^k],\alpha\in(0,1)\\
	&X^{k+1}=X^k+\frac{\Ma^*(y^{k+1})-S^{k+1}}{\mu^k}
}

\p但(2.5)并不容易解得,因此我们考虑使用交替方向法来求解这个方程,即
\equAlign{
	&y^{k+1}=\arg\min_{y,S}L(y,S^k,X^k,\mu^k)\\
	&S^{k+1}=\arg\min_{y,S}L(y^{k+1},S,X^k,\mu^k)
}
$\mu^{k+1}$和$X^{k+1}$的求法不变.

\p首先来求解(2.8)
\equ{L(y,S^k,X^k,\mu^k)=-b^\top y+\Ma(X^k)^\top y-\inprod{X,S^k}+\frac{1}{2\mu^k}\norm{\Ma^*(y)-S^k}^2_F}

记$T=\Ma^*(y)-S^k,\Ma_{ij}=(A_{1_{ij}},\cdots,A_{m_{ij}})\in\Real{m}$,
考虑$\norm{T}^2_F$对y的梯度
\[\norm{T}^2_F=\sum_{i,j}((\sum^m_{k=1}y_kA_{k_{ij}})-S^k_{ij})^2=\sum_{i,j}(\Ma_{ij}^\top y-S^k_{ij})^2\]

则
\equ{
	\begin{aligned}
		\partD{\norm{T}^2_F}{y} & =\partD{\sum_{i,j}(\Ma_{ij}^\top y-S^k_{ij})^2}{y}=\sum_{i,j}\partD{(\Ma_{ij}^\top y-S^k_{ij})^2}{y} \\
		                        & =2\sum_{i,j}(\Ma_{ij}^\top y-S^k_{ij})\Ma_{ij}=2(\sum_{i,j}T_{ij}A_{k_{ij}})=2(\inprod{A_i,T})       \\
		                        & =2\Ma(\Ma^*(y)-S^k)
	\end{aligned}
}
\p由(2.11)可以给出梯度
\equ{
	\partD{L(y,S^k,X^k,\mu^k)}{y}=-b+\Ma(X^k)+\frac{1}{\mu^k}\Ma(\Ma^*(y)-S^k)
}

\p令(2.12)式等于0,可得
\equ{y^{k+1}=\mu^k(b-\Ma(X^k))+\Ma(S^k)}

\p再考虑(2.9).将(2.10)如下变形
\equSplit{
	L(y^{k+1},S,X^k,\mu^k)&=-b^\top y^{k+1}+\Ma(X^k)^\top y^{k+1}-\inprod{X,S}+\frac{1}{2\mu^k}\norm{\Ma^*(y^{k+1})-S}^2_F\\
	&=\frac{1}{2\mu^k}(\norm{S}_F^2-2\inprod{Y,S}+\norm{\Ma^*(y^{k+1})}_F^2)-b^\top y^{k+1}+\Ma(X^k)^\top y^{k+1}\\
}
其中$Y=\Ma^*(y^{k+1})+\mu^kX^k$.

\equ{
	\partD{L(y^{k+1},S,X^k,\mu^k)}{S}=\frac{1}{2\mu^k}(\partD{\norm{S}_F^2}{S}-2\partD{\inprod{Y,S}}{S})=\frac{S-Y}{\mu^k}
}

\p令(2.15)式等于0,则可得$S^{k+1}=Y$,但由约束条件$\norm{S}_2\leq1$,需对$Y$进行修正,即

\equ{S^{k+1}=UDiag(min\{\sigma,1\})V^\top}
其中$Y=UDiag(\sigma)V^\top$.根据$Y$的定义,可以简化$X^{K+1}$的计算方法
\equ{X^{K+1}=\frac{\mu^kX^k+\Ma^*(y^{k+1})-S^{k+1}}{\mu^k}=\frac{Y-S^{k+1}}{\mu^k}}

\p重复进行上述迭代,到目标精度停止,下面给出相应算法

\begin{algo}
	\s\\
	步1\s给出$\mu^0,X^0,y^0,S^0,\epsilon,\alpha$,设置计数器$k=0$;\\
	步2\s若$\frac{\norm{X^{k+1}-X^k}_F}{\max\{1,\norm{X^k}_F\}}\leq\epsilon$,则停止;\\
	步3\s计算$y^{k+1}=\mu^k(b-\Ma(X^k))+\Ma(S^k)$;\\
	步4\s计算$Y=\Ma^*(y^{k+1})+\mu^kX^k$,并计算其SVD:$Y=UDiag(\sigma)V^\top$;\\
	步5\s计算$S^{k+1}=UDiag(min\{\sigma,1\})V^\top$;\\
	步6\s计算$X^{K+1}=\frac{Y-S^{k+1}}{\mu^k}$;\\
	步7\s计算$\mu^{k+1}=\alpha\mu^k$,$k=k+1$,转步2;
\end{algo}

\sect{RBR法}
\p对于问题(1.4),我们可以考虑把它转化为一个半定规划(SDP)问题.

\p一个标准的半定规划问题是
\equSplit{
	\min_{X\in S^n}\s&\inprod{C,X}\\
	\Tst&\Ma(X)=b,X\succ0}
其中$b\in\Real{m},\Ma(X)=(\inprod{A_1,X},\cdots,\inprod{A_m,X})$,$C,A_i\in S^n$,$S^n$是全体对称矩阵.

\p对于一个对称正定矩阵$X\in S^n$,我们可以把它写成分块矩阵的形式
\equ{
	X=\pMa{\xi&y^\top\\ y&B}
}
其中$\xi\in\Real{},y\in\Real{n-1},B\in S^{n-1}$.

\p容易验证的,$X$可以表示为以下形式
\equ{
	X=\pMa{1&y^\top B^{-1}\\0&I}\pMa{\xi-y^\top B^{-1}y&0\\0&B}\pMa{1&0\\B^{-1}y&I}
}
记$(X/B)=\xi-y^\top B^{-1}y$,称为X对于B的Schur补.

\p显然的,\equ{X\succeq0\Leftrightarrow B\succeq0,(X/B)\geq0}

\p约定以下记号
\equ{
	X_{\alpha,\beta}=
	\begin{cases}
		x_{\alpha\beta}
		&\alpha,\beta\in\Real{}\\
		(x_{\alpha\beta_1},\cdots,x_{\alpha\beta_n})
		&\alpha\in\Real{},\beta=\{\beta_1,\cdots,\beta_n\}\\
		(x_{\alpha_1\beta},\cdots,x_{\alpha_m\beta})^\top
		&\alpha=\{\alpha_1,\cdots,\alpha_n\},\beta\in\Real{}\\
		\pMa{
		x_{\alpha_1\beta_1} & \cdots & x_{\alpha_1\beta_n} \\
		\cdots              & \cdots & \cdots              \\
		x_{\alpha_m\beta_1} & \cdots & x_{\alpha_m\beta_n}
		}
		&\alpha=\{\alpha_1,\cdots,\alpha_n\},\beta=\{\beta_1,\cdots,\beta_n\}
	\end{cases}
}
\equ{i^c=\{1,\cdots,n\}\backslash\{i\}=\{1,\cdots,i-1,i+1,\cdots,n\}}

\p令$X=\pMa{\xi&y^\top\\ y&B}=\pMa{X_{i,i}&X_{i,i^c}\\X_{i^c,i}&X_{i^c,i^c}}$,
等号在相差一个初等变化下成立.基于(3.4),
令$i$取遍$\{1,\cdots,n\}$,逐行解如下的SOCP问题来解决SDP问题(3.1)

\equSplit{
	\min_{[\xi;y]\in\Real{n}}\s&\bar{c}^\top\pMa{\xi\\y}\\
	\Tst&\bar{A}\pMa{\xi\\y}=\bar{b},\\
	&(X/B)\geq\delta
}

其中
\equSplit{
	\bar{c}=\pMa{C_{i,i}\\2C{i^c,i}},\s
	\bar{A}=\pMa{A^(1)_{i,i}&2A^(1)_{i,i^c}\\\cdots&\cdots\\A^(m)_{i,i}&A^(m)_{i,i^c}},\s
	\bar{b}=\pMa{b_1-\inprod{A^(1)_{i^c,i^c},B}\\\cdots\\b_m-\inprod{A^(m)_{i^c,i^c},B}}
}
若$X$是半正定的,即$X\succeq0$,取$\delta=0$;若X是正定的,即$X\succ0$,用大于零的数来限制Schur补,取$\delta>0$

\p这种逐行求解的方法就是RBR方法,给出算法
\begin{algo}
	\s\\
	步1\s给出$\delta\geq0,X^1\succeq0,F^0=\inprod{C,X^1},F^1=+\infty,\epsilon>0$,设置计数器$k=1,i=1$;\\
	步2\s若$\frac{\abs{F^{k-1}-F^k}}{\max\{1,\abs{F^{k-1}}\}}\leq\epsilon$,则停止;\\
	步3\s若$i>n$,则令$i=1,X^{k+1}=X^k,k=k+1$,转步2;\\
	步4\s依照定义,给出$X^{k+1}$的$B,\bar{c},\bar{A},\bar{b}$;\\
	步5\s求解问题(3.5),最优解为$(\xi,y)$;\\
	步6\s令$X^{k}_{i,i}=\xi,X^{k}_{i,i^c}=y,X^{k}_{i^c,i}=y^\top$;\\
	步7\s$i=i+1$,转步3;
\end{algo}

\p考虑(3.7)的Powell罚函数
\equ{
	F(X,\theta,\mu)=\bar{c}^\top\pMa{\xi\\y}+\frac{1}{2\mu}\norm{\bar{A}[\xi;y]-\bar{b}-\theta}^2_2
}
其中$\theta\in\Real{m}$,$\mu>0$是给定常数.记$\lambda=\theta+\bar{b}$

\p(3.7)等价于以下问题
\equSplit{
	\min_{X}\s&F(X,\lambda,\mu)=\bar{c}^\top\pMa{\xi\\y}+\frac{1}{2\mu}\norm{\bar{A}[\xi;y]-\lambda}^2_2\\
	\Tst&X\succeq0
}

\p回头考虑(1.4).当$X\in S^n$对称正定,有$\nunorm=tr(X)$,(1.4)d等价于下面的SDP问题
\equSplit{
	\min_X\s&tr(X)=\inprod{E,X}\\
	\Tst&X_{ij}=M_{ij},\forall(i,j)\in\Omega
}
\p当$X\in\Real{p\times q}$不是对称正定的,可以一个更大的对称正定矩阵$W$.当补全了$W$,则$X$自然被补全了(当然,当$X$是对称正定矩阵时,我们也可以这么做)
\equSplit{
	\min_X\s&tr(X)\\
	\Tst&X=\pMa{X_1&W\\W^\top&X_2}\succ0\\
	&W_{ij}=M_{ij},\forall(i,j)\in\Omega
}
其中$X\in S^{n},n=p+q,W_1\in S^p,W_2\in S^q$,$X,W_1,W_2\succ0$.

\p我们主要讨论一般的情况,即问题(3.12).采用RBR法,对于某个i,我们把向量$y$分为两个部分,即
\equ{y=\pMa{y_1\\y_2},\s y_1=X_{\alpha_i,i},y_2=X_{\beta_i,i}}
其中
$\alpha_i=
\begin{cases}
	\{j+p\vert(i,j\in\Omega)\},i\leq p   \\
	\{j\vert(j,i-p)\in\Omega\},p<i\leq n
\end{cases}
,\s
\beta_i=\{1,\cdots,p\}\backslash(\alpha_i\cup\{i\})$,$y_1$是$X$第$i$列除去第$i$行后所有已知元素构成的列向量,$y_2$是$X$第$i$列除去第$i$行后所有未知元素构成的列向量.
\p相应的,$B=\pMa{X_{\alpha_i,\alpha_i}&X_{\alpha_i,\beta_i}\\X_{\beta_i,\alpha_i}&X_{\beta_i,\beta_i}}$,$\s\xi=X_{i,i}$.同时,可以给出$\bar{A}\pMa{\xi\\y},\bar{b}$和$\bar{c}$的显式表达
\equ{
	\bar{b}=\begin{cases}
	(M_{i,\alpha_i-p})^\top,&i\leq p\\
	M_{\alpha_i,i-p}\;,&p<i\leq n
	\end{cases},\s
	\bar{A}\pMa{\xi\\y}=y_1,
	\bar{c}=(1,\overbrace{0,\cdots,0}^{n-1})
}

\p故(3.10)化为以下形式
\equSplit{
	\min\s&\xi+\frac{1}{2\mu}\norm{y_1-\lambda}^2_2\\
	\Tst&\xi-y^\top B^{-1}y\geq0
}



\sect{FPCA法}

\sect{补充知识}
此部分会给出一些阅读本文必备的结论,许多结论将略过证明.
\ssect{奇异值分解}
$\forall X\in\Real{p\times q},X=UDV^\top$,
其中$U\in\Real{p\times p},V\in\Real{q\times q}$
是正交矩阵,
$D=Diag(\sigma)\in\Real{p\times q}$
对角线上为非负实数,其他位置为0的矩阵.

\p这个分解称为\b{奇异值分解(SVD)},
对角矩阵Diag($\sigma$)的每个对角元素$\sigma_i$都是非负的,
称为矩阵的\b{奇异值},
正交矩阵$U$的列向量称为矩阵的\b{左奇异向量},
正交矩阵$V$的列向量称为矩阵的\b{右奇异向量}.

\p有时也认为$X=UDV^\top$,
其中$U\in\Real{p\times r},V\in\Real{q\times r}$
是列正交矩阵,
$D\in\Real{r\times r}$
是非负实对角矩阵,
$r=rank(X)$.
%\[XX^\top=UDV^\top VD^\top U^\top=UDD^\top U^\top\]
%\[X^\top X=VD^\top U^\top UDV^\top=VD^\top DV^\top\]
%\p从上面的式子可以看出,奇异值和特征值还有着密切的联系:\b{(1)}.$U$的列向量(左奇异向量)是 $XX^{\top}$的特征向量;\b{(2)}.$V$的列向量(右奇异向量)是 $X^{\top}X$的特征向量;\b{(3)}.$D$的非零对角元(非零奇异值)是$XX^{\top}$或者$X^{\top}X$的非零特征值的平方根。

\ssect{罚函数}
\p对于约束优化问题:
\equSplit{
	\min_x\s&f(x),\\
	\Tst&c_i(x)=0,\s i=1,\cdots,m_e;\\
	&c_i(x)\geq 0,\s i=m_e+1,\cdots,m
}
\p我们可以通过把约束条件平方后,
乘上一个很大的常数$\mu$,
再加到目标函数上,转化为\b{罚函数}:

\equ{P(x,\mu)=f(x)+\mu \norm{\bar{c}(x)}^2_2}
其中$\bar{c}(x)=(\bar{c}_1(x),\cdots,\bar{c}_m(x))\in\Real{m}$,$\,\bar{c}_i(x)$如下定义:
$\bar{c}_i(x)=
\begin{cases}
	c_i(x),\s i\leq m_e       \\
	\max\{0,c_i(x)\},\s i>m_e \\
\end{cases}
$.
\p当$\mu\rightarrow\infty$,(5.3)和原问题(5.1)等价.
\equ{\min_x\s P(x,\mu)}

\p但在实际中,$\mu$无法取到无穷大,为了克服这一缺点,可以定义\b{增广Lagrange函数}(以下叙述仅考虑等值约束的情况,即$m_e=m$):
\equ{P(x,\lambda,\mu)=f(x)-\lambda^\top c(x)+\frac{1}{2}\mu\norm{c(x)}^2_2}
其中$\lambda,c(x)\in\Real{m},\lambda,\mu$给定.

\p(5.4)等价于以下的\b{Powell罚函数}:
\equ{P(x,\theta,\mu)=f(x)+\frac{1}{2}\mu\norm{c(x)-\theta}^2_2}
其中$c(x),\theta\in\Real{m}$,\,(5.4)与(5.5)只相差一个与$x$无关的常数$\frac{1}{2}\sigma\norm{\theta}^2_2$

\ssect{对偶问题}

\p对于约束优化问题(5.1),它的\b{Langrange函数}为:
\equ{L(x,\mu,\lambda)=f(x)+\sum^{m_e}_{i=1}\mu_ic_i(x)+\sum^{m}_{i=m_e+1}\lambda_ic_i(x)}

\p定义$g(\mu,\lambda)=\inf_xL(x,\mu,\lambda)$,\s通过这个函数我们可以把(5.1)转化为它的对偶问题:
\equSplit{
	\max_{\mu,\lambda}\s g(\mu,\lambda)\\
	\Tst\mu\succeq0\\
}
当满足一定条件(KKT条件)时,(5.7)和(5.1)是等价的.

\ssect{次梯度}

\p$f:E\rightarrow\Real{}$
是一个定义在$\Real{n}$的凸子集$E$上的实值函数,则所有满足
\equ{f(y)-f(x)\geq u(y-x),\forall y\in E}
的向量$u$称为$f$在$x$点的\b{次梯度},
所有这样的$u$构成的集合称为$f$在$x$点的\b{次梯度集}.
\p次梯度集通常用$\partial f(x)$表示
\equ{\partial f(x)=\{u\in\Real{m}\vert f(y)-f(x)\geq u(y-x),\forall y\in E\}}
当函数在$x$点可微,$\partial f(x)=\{f^\prime(x)\}$;当函数在$x$点不可微,$\partial f(x)$是一个非空的紧凸集.

\p对于不可微函数或者不可微点,我们可以采用次梯度替代梯度进行分析.众所周知,一个点是最优解的必要条件是该点梯度为0.相应的,有如下结论:
\begin{theo}
	$x^*$是(5.1)问题的最优解的一个必要条件是:$0\in\partial f(x^*)$.特别的,若$f$是凸函数,则是充要条件.
\end{theo}

\ssect{矩阵范数}
\p以下默认$A\in\Real{m\times n},x\in\Real{n}$,$k=\max\{m,n\},\lambda(A)=(\lambda_1(A),\cdots,\lambda_k(A))\in\Real{k}$是矩阵$A$的奇异值向量,$\lambda_i(A)$是矩阵$A$的第i个奇异值.
\sssect{诱导-p范数}
\p矩阵的\b{诱导-p范数}是将矩阵看作线性算子,由向量的p-范数诱导而来.
\[\norm{A}_p=\sup_{\norm{x}_p\leq1}\norm{Ax}_p\]
\p常见的有\b{2-范数}:
\[\norm{A}_2=\max_i\lambda_i(A)\]

\p\b{1-范数},又称\b{列范数}:
\[\norm{A}_1=\max_{1\leq j\leq n}\sum_{i=1}^m\abs{a_{ij}}\]

\p\b{$\infty$-范数},又称\b{行范数}:
\[\norm{A}_\infty=\max_{1\leq i\leq m}\sum_{j=1}^n\abs{a_{ij}}\]

\sssect{F-范数}
\p矩阵的\b{F-范数}定义类似于向量的2-范数:
\[\norm{A}_F=(\sum_{i,j}a_{ij}^2)^{\frac{1}{2}}\]
\p定义矩阵内积$\inprod{A,X}=tr(A^\top X)$,F-范数与矩阵内积是相容的,可以看作内积诱导的范数,即$\norm{A}_F=\inprod{A,A}^{\frac{1}{2}}$.

\sssect{Schaten-p范数}
\p矩阵的\b{Schaten-p范数}是p-范数应用与矩阵奇异值向量所得的:
\[\norm{A}_p=(\sum_i^k\lambda_i(A)^p)^{\frac{1}{p}}=\norm{\lambda(A)}_p\]

\p矩阵的\b{核范数},又称\b{迹范数},是在$p=1$时Schaten-p范数:
\[\norm{A}_*=\sum_i^k\lambda_i(X)=tr(\sqrt{AA^\top})=\norm{A}_{tr}\]

%\p我们来说明这一点.对$A$做奇异值分解$A=UDV^\top,D=Diag(\sigma)$,设$B=UDU^{-1}$,则$B\in\Real{m\times m}$,相似于一个非负的实对角矩阵,因而是对称正定的.
%\[AA^\top=UDV^\top VD^\top U^\top=UDD^\top U^\top=UDU^{-1}UD^\top U^{-1}=BB^\top=B^2\]
%因此$B=\sqrt{AA^\top}$是well-defined,即有
%\equ{\norm{A}_*=\sum_{i=1}^r\sigma_i=tr(D)=tr(B)=tr(\sqrt{AA^\top})=\norm{A}_{tr}}

\p特别的,当$A$是对称正定的方阵,则有$A=\sqrt{AA^\top}$,即
\[\norm{A}_*=tr(A)\]

\p核范数$\norm{A}_*=\norm{\lambda(A)}_1$,而矩阵的秩$rank(A)=\norm{\lambda(A)}_0$,由此可以看出,核范数与秩的关系就是1-范数与0-范数的关系,故采用核范数来近似秩.

\p核范数是非光滑函数的,次梯度集如下
\equ{\partial\nunorm=\{UV^\top+W\vert U^\top W=0,WV=0,\norm{W}_2\leq1\}}
其中$X=UDV^\top$

\end{document}
